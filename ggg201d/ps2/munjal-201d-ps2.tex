\documentclass[]{article}
\usepackage[T1]{fontenc}
\usepackage{lmodern}
\usepackage{amssymb,amsmath}
\usepackage{ifxetex,ifluatex}
\usepackage{fixltx2e} % provides \textsubscript
% Set line spacing
% use upquote if available, for straight quotes in verbatim environments
\IfFileExists{upquote.sty}{\usepackage{upquote}}{}
\ifnum 0\ifxetex 1\fi\ifluatex 1\fi=0 % if pdftex
  \usepackage[utf8]{inputenc}
\else % if luatex or xelatex
  \ifxetex
    \usepackage{mathspec}
    \usepackage{xltxtra,xunicode}
  \else
    \usepackage{fontspec}
  \fi
  \defaultfontfeatures{Mapping=tex-text,Scale=MatchLowercase}
  \newcommand{\euro}{€}
\fi
% use microtype if available
\IfFileExists{microtype.sty}{\usepackage{microtype}}{}
\usepackage[margin=1in]{geometry}
\usepackage{color}
\usepackage{fancyvrb}
\newcommand{\VerbBar}{|}
\newcommand{\VERB}{\Verb[commandchars=\\\{\}]}
\DefineVerbatimEnvironment{Highlighting}{Verbatim}{commandchars=\\\{\}}
% Add ',fontsize=\small' for more characters per line
\usepackage{framed}
\definecolor{shadecolor}{RGB}{248,248,248}
\newenvironment{Shaded}{\begin{snugshade}}{\end{snugshade}}
\newcommand{\KeywordTok}[1]{\textcolor[rgb]{0.13,0.29,0.53}{\textbf{{#1}}}}
\newcommand{\DataTypeTok}[1]{\textcolor[rgb]{0.13,0.29,0.53}{{#1}}}
\newcommand{\DecValTok}[1]{\textcolor[rgb]{0.00,0.00,0.81}{{#1}}}
\newcommand{\BaseNTok}[1]{\textcolor[rgb]{0.00,0.00,0.81}{{#1}}}
\newcommand{\FloatTok}[1]{\textcolor[rgb]{0.00,0.00,0.81}{{#1}}}
\newcommand{\CharTok}[1]{\textcolor[rgb]{0.31,0.60,0.02}{{#1}}}
\newcommand{\StringTok}[1]{\textcolor[rgb]{0.31,0.60,0.02}{{#1}}}
\newcommand{\CommentTok}[1]{\textcolor[rgb]{0.56,0.35,0.01}{\textit{{#1}}}}
\newcommand{\OtherTok}[1]{\textcolor[rgb]{0.56,0.35,0.01}{{#1}}}
\newcommand{\AlertTok}[1]{\textcolor[rgb]{0.94,0.16,0.16}{{#1}}}
\newcommand{\FunctionTok}[1]{\textcolor[rgb]{0.00,0.00,0.00}{{#1}}}
\newcommand{\RegionMarkerTok}[1]{{#1}}
\newcommand{\ErrorTok}[1]{\textbf{{#1}}}
\newcommand{\NormalTok}[1]{{#1}}
\usepackage{graphicx}
\usepackage{float}
% Redefine \includegraphics so that, unless explicit options are
% given, the image width will not exceed the width of the page.
% Images get their normal width if they fit onto the page, but
% are scaled down if they would overflow the margins.

\makeatletter
\def\ScaleIfNeeded{%
  \ifdim\Gin@nat@width>\linewidth
    \linewidth
  \else
    \Gin@nat@width
  \fi
}
\makeatother
\let\Oldincludegraphics\includegraphics
{%
 \catcode`\@=11\relax%
 \gdef\includegraphics{\@ifnextchar[{\Oldincludegraphics}{\Oldincludegraphics[width=\ScaleIfNeeded]}}%
}%
\ifxetex
  \usepackage[setpagesize=false, % page size defined by xetex
              unicode=false, % unicode breaks when used with xetex
              xetex]{hyperref}
\else
  \usepackage[unicode=true]{hyperref}
\fi
\hypersetup{breaklinks=true,
            bookmarks=true,
            pdfauthor={Gitanshu Munjal},
            pdftitle={Problem Set 2},
            colorlinks=true,
            citecolor=blue,
            urlcolor=blue,
            linkcolor=magenta,
            pdfborder={0 0 0}}
\urlstyle{same}  % don't use monospace font for urls
\setlength{\parindent}{0pt}
\setlength{\parskip}{6pt plus 2pt minus 1pt}
\setlength{\emergencystretch}{3em}  % prevent overfull lines
\setcounter{secnumdepth}{0}

%%% Change title format to be more compact
\usepackage{titling}
\setlength{\droptitle}{-2em}
  \title{Problem Set 2}
  \pretitle{\vspace{\droptitle}\centering\huge}
  \posttitle{\par}
  \author{Gitanshu Munjal}
  \preauthor{\centering\large\emph}
  \postauthor{\par}
  \predate{\centering\large\emph}
  \postdate{\par}
  \date{Saturday, April 25, 2015}




\begin{document}

\maketitle


\section{Problem 1}\label{problem-1}

\subsection{Using the Pairwise FST data construct a neighbor joining
tree including distances for the Igbo, Yoruba, Kongo, Bamoun, Fulani
populations (branches don't have to be perfectly to scale). (15
pts)}\label{using-the-pairwise-fst-data-construct-a-neighbor-joining-tree-including-distances-for-the-igbo-yoruba-kongo-bamoun-fulani-populations-branches-dont-have-to-be-perfectly-to-scale.-15-pts}

\begin{Shaded}
\begin{Highlighting}[]
\NormalTok{################################################}
\CommentTok{# Neighbor Joining Tree}
\NormalTok{################################################}

\CommentTok{# Clear workspace, set working directory, read in Fst matrix (reduced from paper)}
\KeywordTok{rm}\NormalTok{(}\DataTypeTok{list=}\KeywordTok{ls}\NormalTok{())}
\KeywordTok{setwd}\NormalTok{(}\StringTok{"C:}\CharTok{\textbackslash{}\textbackslash{}}\StringTok{Users}\CharTok{\textbackslash{}\textbackslash{}}\StringTok{uglysweaters}\CharTok{\textbackslash{}\textbackslash{}}\StringTok{Desktop}\CharTok{\textbackslash{}\textbackslash{}}\StringTok{GGG201D}\CharTok{\textbackslash{}\textbackslash{}}\StringTok{ProblemSet2}\CharTok{\textbackslash{}\textbackslash{}}\StringTok{"}\NormalTok{)}

\NormalTok{fstmatrix <-}\StringTok{ }\KeywordTok{read.table}\NormalTok{(}\StringTok{"data}\CharTok{\textbackslash{}\textbackslash{}}\StringTok{tab1.txt"}\NormalTok{,}\DataTypeTok{header=}\NormalTok{T,}\DataTypeTok{row.names=}\DecValTok{1}\NormalTok{,}\DataTypeTok{sep=}\StringTok{"}\CharTok{\textbackslash{}t}\StringTok{"}\NormalTok{)}
\NormalTok{fstmatrix <-}\StringTok{ }\KeywordTok{as.matrix}\NormalTok{(fstmatrix)}
\NormalTok{fstmatrix}
\end{Highlighting}
\end{Shaded}

\begin{verbatim}
##         Igbo Yoruba Kongo Bamoun Fulani
## Igbo      NA  0.084 0.282  0.293  3.905
## Yoruba 0.084     NA 0.291  0.318  4.034
## Kongo  0.282  0.291    NA  0.175  3.770
## Bamoun 0.293  0.318 0.175     NA  3.996
## Fulani 3.905  4.034 3.770  3.996     NA
\end{verbatim}

\begin{Shaded}
\begin{Highlighting}[]
\CommentTok{# Find pair with minimum distance (ignore diagnoal NAs)}
\KeywordTok{which}\NormalTok{(fstmatrix ==}\StringTok{ }\KeywordTok{min}\NormalTok{(fstmatrix,}\DataTypeTok{na.rm=}\NormalTok{T),T)}
\end{Highlighting}
\end{Shaded}

\begin{verbatim}
##        row col
## Yoruba   2   1
## Igbo     1   2
\end{verbatim}

\begin{Shaded}
\begin{Highlighting}[]
\CommentTok{# record distance between Igbo and Yoruba}
\NormalTok{y1 <-}\KeywordTok{c}\NormalTok{(}\StringTok{"I-Y"}\NormalTok{,}\KeywordTok{min}\NormalTok{(fstmatrix,}\DataTypeTok{na.rm=}\NormalTok{T))}
\end{Highlighting}
\end{Shaded}

\pagebreak

\begin{Shaded}
\begin{Highlighting}[]
\NormalTok{##########}
\CommentTok{# Step 2}
\NormalTok{##########}

\CommentTok{# Make new matrix with one less row and one less column, set up col and row names}
\NormalTok{fstmatrix2 <-}\StringTok{ }\KeywordTok{matrix}\NormalTok{(}\OtherTok{NA}\NormalTok{,}\KeywordTok{nrow}\NormalTok{(fstmatrix)-}\DecValTok{1}\NormalTok{,}\KeywordTok{ncol}\NormalTok{(fstmatrix)-}\DecValTok{1}\NormalTok{)}
\KeywordTok{rownames}\NormalTok{(fstmatrix2) <-}\StringTok{ }\KeywordTok{c}\NormalTok{(}\StringTok{"I-Y"}\NormalTok{,}\StringTok{"K"}\NormalTok{,}\StringTok{"B"}\NormalTok{,}\StringTok{"F"}\NormalTok{)}
\KeywordTok{colnames}\NormalTok{(fstmatrix2) <-}\StringTok{ }\KeywordTok{c}\NormalTok{(}\StringTok{"I-Y"}\NormalTok{,}\StringTok{"K"}\NormalTok{,}\StringTok{"B"}\NormalTok{,}\StringTok{"F"}\NormalTok{)}

\CommentTok{# Calculate rlevant pairwise distances using first matrix as described in class notes}
\NormalTok{fstmatrix2[}\DecValTok{2}\NormalTok{:}\DecValTok{4}\NormalTok{,}\DecValTok{1}\NormalTok{] <-}\StringTok{ }\NormalTok{(fstmatrix[}\DecValTok{3}\NormalTok{:}\DecValTok{5}\NormalTok{,}\StringTok{"Igbo"}\NormalTok{]+fstmatrix[}\DecValTok{3}\NormalTok{:}\DecValTok{5}\NormalTok{,}\StringTok{"Yoruba"}\NormalTok{])/}\DecValTok{2}
\NormalTok{fstmatrix2[}\DecValTok{2}\NormalTok{:}\DecValTok{4}\NormalTok{,}\DecValTok{2}\NormalTok{] <-}\StringTok{ }\NormalTok{fstmatrix[}\DecValTok{3}\NormalTok{:}\DecValTok{5}\NormalTok{,}\DecValTok{3}\NormalTok{] }\CommentTok{#Distance same as first matrix}
\NormalTok{fstmatrix2[}\DecValTok{2}\NormalTok{:}\DecValTok{4}\NormalTok{,}\DecValTok{3}\NormalTok{] <-}\StringTok{ }\NormalTok{fstmatrix[}\DecValTok{3}\NormalTok{:}\DecValTok{5}\NormalTok{,}\DecValTok{4}\NormalTok{] }\CommentTok{#Distance same as first matrix}
\NormalTok{fstmatrix2[}\DecValTok{2}\NormalTok{:}\DecValTok{4}\NormalTok{,}\DecValTok{4}\NormalTok{] <-}\StringTok{ }\NormalTok{fstmatrix[}\DecValTok{3}\NormalTok{:}\DecValTok{5}\NormalTok{,}\DecValTok{5}\NormalTok{] }\CommentTok{#Distance same as first matrix}
\NormalTok{fstmatrix2}
\end{Highlighting}
\end{Shaded}

\begin{verbatim}
##        I-Y     K     B     F
## I-Y     NA    NA    NA    NA
## K   0.2865    NA 0.175 3.770
## B   0.3055 0.175    NA 3.996
## F   3.9695 3.770 3.996    NA
\end{verbatim}

\begin{Shaded}
\begin{Highlighting}[]
\CommentTok{# Find pair with minimum distance (ignore diagnoal NAs)}
\KeywordTok{which}\NormalTok{(fstmatrix2 ==}\StringTok{ }\KeywordTok{min}\NormalTok{(fstmatrix2,}\DataTypeTok{na.rm=}\NormalTok{T),T)}
\end{Highlighting}
\end{Shaded}

\begin{verbatim}
##   row col
## B   3   2
## K   2   3
\end{verbatim}

\begin{Shaded}
\begin{Highlighting}[]
\CommentTok{# record distance between Bamoun and Kongo}
\NormalTok{y2 <-}\KeywordTok{c}\NormalTok{(}\StringTok{"B-K"}\NormalTok{,}\KeywordTok{min}\NormalTok{(fstmatrix2,}\DataTypeTok{na.rm=}\NormalTok{T))}

\NormalTok{##########}
\CommentTok{# Step 3}
\NormalTok{##########}

\CommentTok{# Make new matrix with one less row and one less column, set up col and row names}
\NormalTok{fstmatrix3 <-}\StringTok{ }\KeywordTok{matrix}\NormalTok{(}\OtherTok{NA}\NormalTok{,}\KeywordTok{nrow}\NormalTok{(fstmatrix2)-}\DecValTok{1}\NormalTok{,}\KeywordTok{ncol}\NormalTok{(fstmatrix2)-}\DecValTok{1}\NormalTok{)}
\KeywordTok{rownames}\NormalTok{(fstmatrix3) <-}\StringTok{ }\KeywordTok{c}\NormalTok{(}\StringTok{"I-Y"}\NormalTok{,}\StringTok{"B-K"}\NormalTok{,}\StringTok{"Fulani"}\NormalTok{)}
\KeywordTok{colnames}\NormalTok{(fstmatrix3) <-}\StringTok{ }\KeywordTok{c}\NormalTok{(}\StringTok{"I-Y"}\NormalTok{,}\StringTok{"B-K"}\NormalTok{,}\StringTok{"Fulani"}\NormalTok{)}

\CommentTok{# Calculate rlevant pairwise distances using first matrix as described in class notes}
\NormalTok{fstmatrix3[}\StringTok{"I-Y"}\NormalTok{,}\DecValTok{2}\NormalTok{] <-}\StringTok{ }\NormalTok{(fstmatrix[}\StringTok{"Kongo"}\NormalTok{,}\StringTok{"Igbo"}\NormalTok{] +}\StringTok{ }\NormalTok{fstmatrix[}\StringTok{"Kongo"}\NormalTok{,}\StringTok{"Yoruba"}\NormalTok{] +}\StringTok{ }
\StringTok{                          }\NormalTok{fstmatrix[}\StringTok{"Bamoun"}\NormalTok{,}\StringTok{"Igbo"}\NormalTok{] +fstmatrix[}\StringTok{"Bamoun"}\NormalTok{,}\StringTok{"Yoruba"}\NormalTok{])/}\DecValTok{4}
\NormalTok{fstmatrix3[}\StringTok{"B-K"}\NormalTok{,}\DecValTok{1}\NormalTok{] <-}\StringTok{ }\NormalTok{(fstmatrix[}\StringTok{"Kongo"}\NormalTok{,}\StringTok{"Igbo"}\NormalTok{] +}\StringTok{ }\NormalTok{fstmatrix[}\StringTok{"Kongo"}\NormalTok{,}\StringTok{"Yoruba"}\NormalTok{] +}\StringTok{ }
\StringTok{                          }\NormalTok{fstmatrix[}\StringTok{"Bamoun"}\NormalTok{,}\StringTok{"Igbo"}\NormalTok{] +fstmatrix[}\StringTok{"Bamoun"}\NormalTok{,}\StringTok{"Yoruba"}\NormalTok{])/}\DecValTok{4}
\NormalTok{fstmatrix3[}\StringTok{"Fulani"}\NormalTok{,}\DecValTok{1}\NormalTok{] <-}\StringTok{ }\NormalTok{(fstmatrix[}\StringTok{"Fulani"}\NormalTok{,}\StringTok{"Igbo"}\NormalTok{] +}\StringTok{ }\NormalTok{fstmatrix[}\StringTok{"Fulani"}\NormalTok{,}\StringTok{"Yoruba"}\NormalTok{])/}\DecValTok{2}
\NormalTok{fstmatrix3[}\StringTok{"Fulani"}\NormalTok{,}\DecValTok{2}\NormalTok{] <-}\StringTok{ }\NormalTok{(fstmatrix[}\StringTok{"Fulani"}\NormalTok{,}\StringTok{"Bamoun"}\NormalTok{]+}\StringTok{ }\NormalTok{fstmatrix[}\StringTok{"Fulani"}\NormalTok{,}\StringTok{"Kongo"}\NormalTok{])/}\DecValTok{2}
\end{Highlighting}
\end{Shaded}

\pagebreak

\begin{Shaded}
\begin{Highlighting}[]
\NormalTok{fstmatrix3}
\end{Highlighting}
\end{Shaded}

\begin{verbatim}
##           I-Y   B-K Fulani
## I-Y        NA 0.296     NA
## B-K    0.2960    NA     NA
## Fulani 3.9695 3.883     NA
\end{verbatim}

\begin{Shaded}
\begin{Highlighting}[]
\CommentTok{# Find pair with minimum distance (ignore diagnoal NAs)}
\KeywordTok{which}\NormalTok{(fstmatrix3 ==}\StringTok{ }\KeywordTok{min}\NormalTok{(fstmatrix3,}\DataTypeTok{na.rm=}\NormalTok{T),T)}
\end{Highlighting}
\end{Shaded}

\begin{verbatim}
##     row col
## B-K   2   1
## I-Y   1   2
\end{verbatim}

\begin{Shaded}
\begin{Highlighting}[]
\CommentTok{# record distance between Igbo-Yoruba and Bamoun-Kongo}
\NormalTok{y3 <-}\KeywordTok{c}\NormalTok{(}\StringTok{"I-Y-B-K"}\NormalTok{,}\KeywordTok{min}\NormalTok{(fstmatrix3,}\DataTypeTok{na.rm=}\NormalTok{T))}

\NormalTok{##########}
\CommentTok{# Step 4}
\NormalTok{##########}

\CommentTok{# Make new matrix with one less row and one less column, set up col and row names}
\NormalTok{fstmatrix4 <-}\StringTok{ }\KeywordTok{matrix}\NormalTok{(}\OtherTok{NA}\NormalTok{,}\KeywordTok{nrow}\NormalTok{(fstmatrix3)-}\DecValTok{1}\NormalTok{,}\KeywordTok{ncol}\NormalTok{(fstmatrix3)-}\DecValTok{1}\NormalTok{)}
\KeywordTok{rownames}\NormalTok{(fstmatrix4) <-}\StringTok{ }\KeywordTok{c}\NormalTok{(}\StringTok{"I-Y-B-K"}\NormalTok{,}\StringTok{"Fulani"}\NormalTok{)}
\KeywordTok{colnames}\NormalTok{(fstmatrix4) <-}\StringTok{ }\KeywordTok{c}\NormalTok{(}\StringTok{"I-Y-B-K"}\NormalTok{,}\StringTok{"Fulani"}\NormalTok{)}

\CommentTok{# Calculate rlevant pairwise distances using first matrix as described in class notes}
\NormalTok{fstmatrix4[}\StringTok{"Fulani"}\NormalTok{,}\DecValTok{1}\NormalTok{] <-}\StringTok{ }\NormalTok{(fstmatrix[}\StringTok{"Fulani"}\NormalTok{,}\StringTok{"Igbo"}\NormalTok{] +}\StringTok{ }\NormalTok{fstmatrix[}\StringTok{"Fulani"}\NormalTok{,}\StringTok{"Yoruba"}\NormalTok{] }
                           \NormalTok{+}\StringTok{ }\NormalTok{fstmatrix[}\StringTok{"Fulani"}\NormalTok{,}\StringTok{"Bamoun"}\NormalTok{] +}\StringTok{ }\NormalTok{fstmatrix[}\StringTok{"Fulani"}\NormalTok{,}\StringTok{"Kongo"}\NormalTok{])/}\DecValTok{4}
\NormalTok{fstmatrix4[}\StringTok{"I-Y-B-K"}\NormalTok{,}\DecValTok{2}\NormalTok{] <-}\StringTok{ }\NormalTok{(fstmatrix[}\StringTok{"Fulani"}\NormalTok{,}\StringTok{"Igbo"}\NormalTok{] +}\StringTok{ }\NormalTok{fstmatrix[}\StringTok{"Fulani"}\NormalTok{,}\StringTok{"Yoruba"}\NormalTok{] }
                           \NormalTok{+}\StringTok{ }\NormalTok{fstmatrix[}\StringTok{"Fulani"}\NormalTok{,}\StringTok{"Bamoun"}\NormalTok{] +}\StringTok{ }\NormalTok{fstmatrix[}\StringTok{"Fulani"}\NormalTok{,}\StringTok{"Kongo"}\NormalTok{])/}\DecValTok{4}
\NormalTok{fstmatrix4}
\end{Highlighting}
\end{Shaded}

\begin{verbatim}
##         I-Y-B-K  Fulani
## I-Y-B-K      NA 3.92625
## Fulani  3.92625      NA
\end{verbatim}

\begin{Shaded}
\begin{Highlighting}[]
\CommentTok{# Find pair with minimum distance (ignore diagnoal NAs)}
\KeywordTok{which}\NormalTok{(fstmatrix4 ==}\StringTok{ }\KeywordTok{min}\NormalTok{(fstmatrix4,}\DataTypeTok{na.rm=}\NormalTok{T),T)}
\end{Highlighting}
\end{Shaded}

\begin{verbatim}
##         row col
## Fulani    2   1
## I-Y-B-K   1   2
\end{verbatim}

\begin{Shaded}
\begin{Highlighting}[]
\CommentTok{# record distance between Igbo-Yoruba-Bamoun-Kongo and Fulani}
\NormalTok{y4 <-}\KeywordTok{c}\NormalTok{(}\StringTok{"I-Y-B-K-F"}\NormalTok{,}\KeywordTok{min}\NormalTok{(fstmatrix4,}\DataTypeTok{na.rm=}\NormalTok{T))}

\NormalTok{y <-}\StringTok{ }\KeywordTok{rbind}\NormalTok{(y1,y2,y3,y4)}
\KeywordTok{colnames}\NormalTok{(y) <-}\StringTok{ }\KeywordTok{c}\NormalTok{(}\StringTok{"NeighborsJoined"}\NormalTok{,}\StringTok{"TiptoMRCADistance"}\NormalTok{)}
\end{Highlighting}
\end{Shaded}

\pagebreak

\begin{Shaded}
\begin{Highlighting}[]
\KeywordTok{data.frame}\NormalTok{(y)}
\end{Highlighting}
\end{Shaded}

\begin{verbatim}
##    NeighborsJoined TiptoMRCADistance
## y1             I-Y             0.084
## y2             B-K             0.175
## y3         I-Y-B-K             0.296
## y4       I-Y-B-K-F           3.92625
\end{verbatim}

The tree based on the above presented data looks as follows:

\begin{figure}[H]
\centering
\includegraphics[width=0.8\textwidth]{njtree.jpg}
\caption{\label{fig:njtree}Neighbor Joining Tree based on data from Bryc et. al (2009).}
\end{figure}

\subsection{Describe how you could use bootstrapping to assess the
relationships presented in your tree. (5
pts)}\label{describe-how-you-could-use-bootstrapping-to-assess-the-relationships-presented-in-your-tree.-5-pts}

We could pull the genotype data used for the study and resample n (where
n = number of individuals used in the study from a given sub-population)
individuals from each of the subpopulations with replacement multiple
(say 1000) time. These data could then be used to generate trees as
discussed in part a. We could then compare our tree from part a to these
bootstrapped trees and assess the relationships in our tree by counting
how many times a given relationship from our tree in part a occurs
equivalently in the bootstrapped set of trees. Considering extreme
findings for the sake of simplicity, if a relation is found to be the
same 999 times one could be pretty confident in it compared to one that
is found to be the same say only 30 out of 1000 times.

\begin{center}
\line(1,0){250}
\end{center}

\pagebreak

\section{Problem 2}\label{problem-2}


\subsection{Do these results indicate that the individual is exactly 1/3
Scandinavian? In your answer, describe how maximum likelihood is used to
estimate admixture across multiple loci. (8
pts)}\label{do-these-results-indicate-that-the-individual-is-exactly-13-scandinavian-in-your-answer-describe-how-maximum-likelihood-is-used-to-estimate-admixture-across-multiple-loci.-8-pts}

The admixture model underlying the presented figure uses a maximum
likelihood framework to make inferences. Before we interpret the
presented data, lets take a look at the underlying analyses.

In a simple case, consider an individual originating from a single
population. Genetic assignment may then be realized by genotyping that
individual and determining the likelihood of the genotype under
different populations based on allele frequencies at the genotyped loci
in each of the populations under consideration. The population that
presents the maximum-likelihood following such a calculation can naively
be considered the population of origin. Under our simple case of origin
from one population, we expect this signal to be rather strong (as shown
in figure 1 for example).

\begin{figure}[H]
\centering
\includegraphics[width=0.6\textwidth]{fig1.jpg}
\caption{\label{fig:njtree}Likelihood Versus Subpopulation. Looks like the genotype/individual very likely came from subpopulation 4.}
\end{figure}

Now consider an individual with hybrid origins. The maximum-likelihood
signal, as discussed above, from such an individual might not be as
defining as presented in figure 1 because of the underlying hybrid
origins (figure 2 for example). Under such a scenario, one expands the
maximum-likelihood framework to account for admixture.

\begin{figure}[H]
\centering
\includegraphics[width=0.6\textwidth]{fig2.jpg}
\caption{\label{fig:njtree}Likelihood Versus Subpopulation. Single underlying subpopulation of origin unlikely, possibly admixed genotype/individual}
\end{figure}

For simplicity, say we have a good suspicion (because sadly for this
hypotheical species, there are only two populations! other reasons could
be based in biology, geography, etc.) that the individual is a hybrid
between two populations. Under such a scenario, we expect a fraction of
this individual's genome ($\theta$) to originate from population 1 and
the remainder ($1-\theta$) from population 2. Based on this and a
knowledge of allele frequencies at genotyped loci in the original two
populations, we can develop probabilistic expectations for each
genotyped loci as shown follows (for a single locus and two populations
s1 and s2).:

\[P(GenotyeObserved\ |\ \theta)\ =\ P[both\ alleles\ from\ s1]\ +\ P[allele1\ from\ s1\ \&\ allele2\ from\ s2]\ +\]
\[\quad \quad \quad \quad \quad \quad \quad \quad \quad \quad \quad \quad \quad \quad P[allele1\ from\ s2\ \&\ allele2\ from\ s1]\ +\ [both\ alleles\ from\ s1]\]
\linebreak
\[\quad \quad \quad =\ (\theta*f_{s1})\ \ \ (\theta*f_{s1})\ +\ (\theta*f_{s1})\ \ \ ((1-\theta)*f_{s2}))\ +\]
\[\quad \quad \quad \quad \quad \quad \quad \quad \quad \quad ((1-\theta)*f_{s2}) \ \ (\theta*f_{s1})\ +\ ((1-\theta)*f_{s2})\ \ \ ((1-\theta)*f_{s2})\]
where,\newline
$\theta$ is the fraction of the genome originating from subpopulation 1
and $f_{s1}$ the frequency of the observed allele in subpopulation 1.

For a multi-locus genotype, the probability of the total genotype
observed can then be calculated simply by multiplying the probabilities
at each locus and can be represented by the following equation:
\[\prod_{i=1}^{n}P(G_{i}\ |\ \theta)\] where,\newline
$G$ is the genotype at the $i^{th}$ locus.

Next, we vary the value of our parameter of interest ($\theta$;
range:0-1) to maximize the above likelihood. The value of $\theta$ that
results in the maximum likelihood is then our best candidate for the
fraction of the genome originating from population
1.\textbf{These values are what is reported in the data shown}. Given
the large number of loci used for determining the presented data, we can
be sure that \textbf{it is very likely that this individual has 1/3
Scandinavian ancestry.}

\subsection{What is one method you could use to assess confidence in
these admixture estimates? (6
pts)}\label{what-is-one-method-you-could-use-to-assess-confidence-in-these-admixture-estimates-6-pts}

For thin slices of $\theta$ (say 0 to 1 in steps of 0.001 for example),
I would determine the total genotype probability as described above,
determine the variance of the determined probability distribution, and
build a confidence interval around it. I would also be interested in
visualizing and maybe reinforcing the findings by doing a PCA with the
SNP set of the individual.
\pagebreak

\subsection{Why is accounting for genetic ancestry useful for studying
the genetic basis of complex traits (such as GWAS)? (6
pts)}\label{why-is-accounting-for-genetic-ancestry-useful-for-studying-the-genetic-basis-of-complex-traits-such-as-gwas-6-pts}

Accounting for genetic ancestry is useful in dissecting the genetic
basis of complex traits as association studies based on diversity panels
with uncharacterized population structure are extremely vulnerable to
spurious associations leading to an inflation of false positives.

Diversity panels composed of mixed and/or admixed individuals with
unknown proportions of multiple genetic origins could give rise to
linkage disequilibrium between unlinked loci resulting in an inflation
of false positive associtations that are not actually involved in
phenotypic variation for the trait being investigated \cite{mezmouk}.

In case-control type disease association studies, spurious associations
are likely if the frequency of disease varies across populations under
consideration \cite{pritchard}. Consider an extreme scenario where 75\%
of the diseased individuals cases come from one population and two
populations are under consideration for the disease association study.
Under such a situation, one would expect to find lots of false positive
alleles that associate with the phenotype purely because of their
enriched frequency in one population. 

\bibliographystyle{acm}
\bibliography{bibi}

\begin{center}
\line(1,0){250}
\end{center}

\pagebreak

\section{Problem 3}\label{problem-3}

\subsection{How is genetic drift is affected by inital allele frequency
and population
size?}\label{how-is-genetic-drift-is-affected-by-inital-allele-frequency-and-population-size}

\subsection{Defining a function}\label{defining-a-function}

Based on the premise that expected change in allele frequency
($\Delta_{expected}^f$) for a given locus in a population of size n is
equal to the summation from 0 to n of the product of the binomial
probability of a given change and the absolute value of the magnitude of
change in frequency over a generation, I define an R function that takes
initial frequency and population size as input parameters to return the
expected change in allele frequency as follows:

\begin{Shaded}
\begin{Highlighting}[]
\NormalTok{efreqchange <-}\StringTok{ }\NormalTok{function(freqgen0,popsize)\{}
  
  \CommentTok{#empty matrix with 6 columns}
  \NormalTok{plotmatrix <-}\StringTok{ }\KeywordTok{matrix}\NormalTok{(,,}\DecValTok{6}\NormalTok{)                                 }
  \KeywordTok{colnames}\NormalTok{(plotmatrix) <-}\StringTok{ }\KeywordTok{c}\NormalTok{(}\StringTok{"nchosen"}\NormalTok{,}\StringTok{"freqgen0"}\NormalTok{,}\StringTok{"dbinom"}\NormalTok{,}\StringTok{"freqgen1"}\NormalTok{,}
                            \StringTok{"deltaf"}\NormalTok{,}\StringTok{"weighted"}\NormalTok{)}
  
  \CommentTok{#column 1 = chosen n, }
  \CommentTok{#column 2 = initial frequency}
  \CommentTok{#column 3 = binomial probabilty of popsize choose n}
  \CommentTok{#column 4 = frequency in next generation = n/popsize}
  \NormalTok{for(nchosen in }\DecValTok{0}\NormalTok{:popsize)\{}
      
    \NormalTok{roww <-}\StringTok{ }\KeywordTok{c}\NormalTok{(nchosen, freqgen0, }\KeywordTok{dbinom}\NormalTok{(nchosen,}\DataTypeTok{size=}\NormalTok{popsize,freqgen0),}
              \NormalTok{nchosen/popsize,}\OtherTok{NA}\NormalTok{,}\OtherTok{NA}\NormalTok{)}
    \NormalTok{plotmatrix <-}\StringTok{ }\KeywordTok{rbind}\NormalTok{(plotmatrix,roww)}
  \NormalTok{\}}
  
  \CommentTok{#column 5 = delta frequency = nextgenfrequency - initialfrequency}
  \CommentTok{#remove rows with NA by ordering and removing end of matrix}
  \NormalTok{plotmatrix <-}\StringTok{ }\NormalTok{plotmatrix[}\KeywordTok{order}\NormalTok{(plotmatrix[,}\DecValTok{2}\NormalTok{]),]}
  \NormalTok{plotmatrix[,}\DecValTok{5}\NormalTok{] <-}\StringTok{ }\NormalTok{plotmatrix[,}\DecValTok{4}\NormalTok{] -plotmatrix[,}\DecValTok{2}\NormalTok{]}
  \NormalTok{plotmatrix <-}\StringTok{ }\NormalTok{plotmatrix[}\DecValTok{1}\NormalTok{:}\KeywordTok{nrow}\NormalTok{(plotmatrix)-}\DecValTok{1}\NormalTok{,]}
  
  
  \CommentTok{#change delta to absolute delta}
  \NormalTok{for(i in }\DecValTok{1}\NormalTok{:}\KeywordTok{nrow}\NormalTok{(plotmatrix))\{}
    \NormalTok{if(plotmatrix[i,}\DecValTok{5}\NormalTok{]<}\DecValTok{0}\NormalTok{)\{}
      \NormalTok{plotmatrix[i,}\DecValTok{5}\NormalTok{] <-}\StringTok{ }\NormalTok{-(plotmatrix[i,}\DecValTok{5}\NormalTok{])}
    \NormalTok{\}}
    \NormalTok{else\{plotmatrix[i,}\DecValTok{5}\NormalTok{] <-}\StringTok{ }\NormalTok{(plotmatrix[i,}\DecValTok{5}\NormalTok{])\}}
  \NormalTok{\}}
  
  \CommentTok{#column 6 = deltaf * binomprobability}
  \NormalTok{plotmatrix[,}\DecValTok{6}\NormalTok{] <-}\StringTok{ }\NormalTok{plotmatrix[,}\DecValTok{5}\NormalTok{]*plotmatrix[,}\DecValTok{3}\NormalTok{]}
  
  \CommentTok{#expected change in allele frequency based on premise}
  \NormalTok{e <-}\StringTok{ }\KeywordTok{sum}\NormalTok{(plotmatrix[,}\DecValTok{6}\NormalTok{])}
  \KeywordTok{return}\NormalTok{(e)}
  
\NormalTok{\}}
\end{Highlighting}
\end{Shaded}

\subsection{Using R or Excel, create a plot that shows how expected
change in allele frequency for allele A changes depending on the
frequency of A in generation 1 in populations of size of 2N=10 and
2N=100. (10
pts)}\label{using-r-or-excel-create-a-plot-that-shows-how-expected-change-in-allele-frequency-for-allele-a-changes-depending-on-the-frequency-of-a-in-generation-1-in-populations-of-size-of-2n10-and-2n100.-10-pts}

\begin{Shaded}
\begin{Highlighting}[]
\NormalTok{popsize <-}\StringTok{ }\DecValTok{10}
\NormalTok{freqchange <-}\StringTok{ }\KeywordTok{matrix}\NormalTok{(}\OtherTok{NA}\NormalTok{,}\DecValTok{11}\NormalTok{,}\DecValTok{2}\NormalTok{)}
\KeywordTok{colnames}\NormalTok{(freqchange) <-}\StringTok{ }\KeywordTok{c}\NormalTok{(}\StringTok{"InitialFrequency"} \NormalTok{, }\StringTok{"ExpectedChange"}\NormalTok{)}

\NormalTok{for(freqgen0 in }\KeywordTok{seq}\NormalTok{(}\DecValTok{0}\NormalTok{,}\DecValTok{1}\NormalTok{,}\FloatTok{0.1}\NormalTok{))\{}
  
  \NormalTok{y <-}\StringTok{ }\KeywordTok{efreqchange}\NormalTok{(freqgen0,popsize)}
  
  \NormalTok{freqchange[(}\DecValTok{10}\NormalTok{*freqgen0)+}\DecValTok{1}\NormalTok{,}\DecValTok{1}\NormalTok{] <-}\StringTok{ }\NormalTok{freqgen0}
  \NormalTok{freqchange[(}\DecValTok{10}\NormalTok{*freqgen0)+}\DecValTok{1}\NormalTok{,}\DecValTok{2}\NormalTok{] <-}\StringTok{ }\NormalTok{y}
\NormalTok{\}}

\NormalTok{(freqchange <-}\StringTok{ }\KeywordTok{data.frame}\NormalTok{(freqchange))}
\end{Highlighting}
\end{Shaded}

\begin{verbatim}
##    InitialFrequency ExpectedChange
## 1               0.0     0.00000000
## 2               0.1     0.06973569
## 3               0.2     0.09663676
## 4               0.3     0.11206773
## 5               0.4     0.12039487
## 6               0.5     0.12304688
## 7               0.6     0.12039487
## 8               0.7     0.11206773
## 9               0.8     0.09663676
## 10              0.9     0.06973569
## 11              1.0     0.00000000
\end{verbatim}

\begin{Shaded}
\begin{Highlighting}[]
\KeywordTok{library}\NormalTok{(ggplot2)}
\KeywordTok{ggplot}\NormalTok{(}\DataTypeTok{data=}\NormalTok{freqchange,}\KeywordTok{aes}\NormalTok{(freqchange[,}\DecValTok{1}\NormalTok{],freqchange[,}\DecValTok{2}\NormalTok{])) +}\StringTok{ }
\StringTok{        }\KeywordTok{geom_line}\NormalTok{(}\DataTypeTok{size=}\DecValTok{1}\NormalTok{) +}\StringTok{ }
\StringTok{        }\KeywordTok{labs}\NormalTok{(}\DataTypeTok{y=}\StringTok{"Expected Change in Frequency"}\NormalTok{,}\DataTypeTok{x=}\StringTok{"Initial Frequency"}\NormalTok{)}
\end{Highlighting}
\end{Shaded}

\includegraphics{munjal-201d-ps2_files/figure-latex/unnamed-chunk-6-1.pdf}

\begin{Shaded}
\begin{Highlighting}[]
\NormalTok{popsize <-}\StringTok{ }\DecValTok{100}
\NormalTok{freqchange <-}\StringTok{ }\KeywordTok{matrix}\NormalTok{(}\OtherTok{NA}\NormalTok{,}\DecValTok{11}\NormalTok{,}\DecValTok{2}\NormalTok{)}
\KeywordTok{colnames}\NormalTok{(freqchange) <-}\StringTok{ }\KeywordTok{c}\NormalTok{(}\StringTok{"InitialFrequency"} \NormalTok{, }\StringTok{"ExpectedChange"}\NormalTok{)}

\NormalTok{for(freqgen0 in }\KeywordTok{seq}\NormalTok{(}\DecValTok{0}\NormalTok{,}\DecValTok{1}\NormalTok{,}\FloatTok{0.1}\NormalTok{))\{}
  
  \NormalTok{y <-}\StringTok{ }\KeywordTok{efreqchange}\NormalTok{(freqgen0,popsize)}
  
  \NormalTok{freqchange[(}\DecValTok{10}\NormalTok{*freqgen0)+}\DecValTok{1}\NormalTok{,}\DecValTok{1}\NormalTok{] <-}\StringTok{ }\NormalTok{freqgen0}
  \NormalTok{freqchange[(}\DecValTok{10}\NormalTok{*freqgen0)+}\DecValTok{1}\NormalTok{,}\DecValTok{2}\NormalTok{] <-}\StringTok{ }\NormalTok{y}
\NormalTok{\}}


\NormalTok{(freqchange <-}\StringTok{ }\KeywordTok{data.frame}\NormalTok{(freqchange))}
\end{Highlighting}
\end{Shaded}

\begin{verbatim}
##    InitialFrequency ExpectedChange
## 1               0.0     0.00000000
## 2               0.1     0.02373576
## 3               0.2     0.03177607
## 4               0.3     0.03644922
## 5               0.4     0.03898519
## 6               0.5     0.03979462
## 7               0.6     0.03898519
## 8               0.7     0.03644922
## 9               0.8     0.03177607
## 10              0.9     0.02373576
## 11              1.0     0.00000000
\end{verbatim}

\begin{Shaded}
\begin{Highlighting}[]
\KeywordTok{library}\NormalTok{(ggplot2)}
\KeywordTok{ggplot}\NormalTok{(}\DataTypeTok{data=}\NormalTok{freqchange,}\KeywordTok{aes}\NormalTok{(freqchange[,}\DecValTok{1}\NormalTok{],freqchange[,}\DecValTok{2}\NormalTok{])) +}\StringTok{ }
\StringTok{        }\KeywordTok{geom_line}\NormalTok{(}\DataTypeTok{size=}\DecValTok{1}\NormalTok{) +}\StringTok{ }
\StringTok{        }\KeywordTok{labs}\NormalTok{(}\DataTypeTok{y=}\StringTok{"Expected Change in Frequency"}\NormalTok{,}\DataTypeTok{x=}\StringTok{"Initial Frequency"}\NormalTok{)}
\end{Highlighting}
\end{Shaded}

\includegraphics{munjal-201d-ps2_files/figure-latex/unnamed-chunk-6-2.pdf}

\subsection{Using R or Excel, create a plot that shows how expected
change in allele frequency for allele A changes depending on population
size, starting with an allele frequency of f(A)=0.5 (10
pts)}\label{using-r-or-excel-create-a-plot-that-shows-how-expected-change-in-allele-frequency-for-allele-a-changes-depending-on-population-size-starting-with-an-allele-frequency-of-fa0.5-10-pts}

\begin{Shaded}
\begin{Highlighting}[]
\NormalTok{freqgen0 <-}\FloatTok{0.5}
\NormalTok{freqchange <-}\StringTok{ }\KeywordTok{matrix}\NormalTok{(}\OtherTok{NA}\NormalTok{,}\DecValTok{11}\NormalTok{,}\DecValTok{2}\NormalTok{)}
\KeywordTok{colnames}\NormalTok{(freqchange) <-}\StringTok{ }\KeywordTok{c}\NormalTok{(}\StringTok{"PopSize"} \NormalTok{, }\StringTok{"ExpectedChange"}\NormalTok{)}
\NormalTok{for(popsize in }\KeywordTok{seq}\NormalTok{(}\DecValTok{10}\NormalTok{,}\DecValTok{100}\NormalTok{,}\DecValTok{10}\NormalTok{))\{}
  
  \NormalTok{y <-}\StringTok{ }\KeywordTok{efreqchange}\NormalTok{(freqgen0,popsize)}
  \NormalTok{freqchange[popsize/}\DecValTok{10}\NormalTok{,}\DecValTok{1}\NormalTok{] <-}\StringTok{ }\NormalTok{popsize}
  \NormalTok{freqchange[popsize/}\DecValTok{10}\NormalTok{,}\DecValTok{2}\NormalTok{] <-}\StringTok{ }\NormalTok{y}
\NormalTok{\}}

\NormalTok{(freqchange <-}\StringTok{ }\KeywordTok{data.frame}\NormalTok{(}\KeywordTok{na.omit}\NormalTok{(freqchange)))}
\end{Highlighting}
\end{Shaded}

\begin{verbatim}
##    PopSize ExpectedChange
## 1       10     0.12304688
## 2       20     0.08809853
## 3       30     0.07223222
## 4       40     0.06268534
## 5       50     0.05613759
## 6       60     0.05128909
## 7       70     0.04751274
## 8       80     0.04446394
## 9       90     0.04193556
## 10     100     0.03979462
\end{verbatim}

\begin{Shaded}
\begin{Highlighting}[]
\KeywordTok{library}\NormalTok{(ggplot2)}
\KeywordTok{ggplot}\NormalTok{(}\DataTypeTok{data=}\NormalTok{freqchange,}\KeywordTok{aes}\NormalTok{(freqchange[,}\DecValTok{1}\NormalTok{],freqchange[,}\DecValTok{2}\NormalTok{])) +}\StringTok{ }
\StringTok{        }\KeywordTok{geom_line}\NormalTok{(}\DataTypeTok{size=}\DecValTok{1}\NormalTok{) +}\StringTok{ }
\StringTok{        }\KeywordTok{labs}\NormalTok{(}\DataTypeTok{y=}\StringTok{"Expected Change in Frequency"}\NormalTok{,}\DataTypeTok{x=}\StringTok{"Population Size"}\NormalTok{)}
\end{Highlighting}
\end{Shaded}

\includegraphics{munjal-201d-ps2_files/figure-latex/unnamed-chunk-7-1.pdf}

\begin{center}
\line(1,0){250}
\end{center}

\section{Problem 4}\label{problem-4}

\subsection{Sketch a neighbor joining tree based on the number of
nucleotide differences for these three species. (5
pts)}\label{sketch-a-neighbor-joining-tree-based-on-the-number-of-nucleotide-differences-for-these-three-species.-5-pts}

\begin{Shaded}
\begin{Highlighting}[]
\NormalTok{###########}
\CommentTok{# Step 1}
\NormalTok{###########}

\CommentTok{#matrix with per site differences data}
\NormalTok{divmatrix <-}\StringTok{ }\KeywordTok{matrix}\NormalTok{(}\OtherTok{NA}\NormalTok{,}\DecValTok{3}\NormalTok{,}\DecValTok{3}\NormalTok{)}
\KeywordTok{colnames}\NormalTok{(divmatrix) <-}\StringTok{ }\KeywordTok{c}\NormalTok{(}\StringTok{"A"}\NormalTok{,}\StringTok{"B"}\NormalTok{,}\StringTok{"C"}\NormalTok{)}
\KeywordTok{rownames}\NormalTok{(divmatrix) <-}\StringTok{ }\KeywordTok{c}\NormalTok{(}\StringTok{"A"}\NormalTok{,}\StringTok{"B"}\NormalTok{,}\StringTok{"C"}\NormalTok{)}
\NormalTok{divmatrix[,}\DecValTok{1}\NormalTok{] <-}\StringTok{ }\KeywordTok{c}\NormalTok{(}\OtherTok{NA}\NormalTok{,}\DecValTok{6}\NormalTok{,}\DecValTok{14}\NormalTok{)/}\DecValTok{1000}
\NormalTok{divmatrix[,}\DecValTok{2}\NormalTok{] <-}\StringTok{ }\KeywordTok{c}\NormalTok{(}\DecValTok{6}\NormalTok{,}\OtherTok{NA}\NormalTok{,}\DecValTok{17}\NormalTok{)/}\DecValTok{1000}
\NormalTok{divmatrix[,}\DecValTok{3}\NormalTok{] <-}\StringTok{ }\KeywordTok{c}\NormalTok{(}\DecValTok{14}\NormalTok{,}\DecValTok{17}\NormalTok{,}\OtherTok{NA}\NormalTok{)/}\DecValTok{1000}
\NormalTok{divmatrix}
\end{Highlighting}
\end{Shaded}

\begin{verbatim}
##       A     B     C
## A    NA 0.006 0.014
## B 0.006    NA 0.017
## C 0.014 0.017    NA
\end{verbatim}

\begin{Shaded}
\begin{Highlighting}[]
\CommentTok{#find pair with minimum per site differences}
\KeywordTok{which}\NormalTok{(divmatrix ==}\StringTok{ }\KeywordTok{min}\NormalTok{(divmatrix,}\DataTypeTok{na.rm=}\NormalTok{T),T)}
\end{Highlighting}
\end{Shaded}

\begin{verbatim}
##   row col
## B   2   1
## A   1   2
\end{verbatim}

\begin{Shaded}
\begin{Highlighting}[]
\CommentTok{#record number of per site differences}
\NormalTok{y1 <-}\StringTok{ }\KeywordTok{c}\NormalTok{(}\StringTok{"A-B"}\NormalTok{,}\KeywordTok{min}\NormalTok{(divmatrix,}\DataTypeTok{na.rm=}\NormalTok{T))}

\NormalTok{###########}
\CommentTok{# Step 2}
\NormalTok{###########}

\CommentTok{#matrix reduced from step 1}
\NormalTok{divmatrix2 <-}\StringTok{ }\KeywordTok{matrix}\NormalTok{(}\OtherTok{NA}\NormalTok{,}\DecValTok{2}\NormalTok{,}\DecValTok{2}\NormalTok{)}
\KeywordTok{colnames}\NormalTok{(divmatrix2) <-}\StringTok{ }\KeywordTok{c}\NormalTok{(}\StringTok{"A-B"}\NormalTok{,}\StringTok{"C"}\NormalTok{)}
\KeywordTok{rownames}\NormalTok{(divmatrix2) <-}\StringTok{ }\KeywordTok{c}\NormalTok{(}\StringTok{"A-B"}\NormalTok{,}\StringTok{"C"}\NormalTok{)}
\NormalTok{divmatrix2[,}\DecValTok{1}\NormalTok{] <-}\StringTok{ }\KeywordTok{c}\NormalTok{(}\OtherTok{NA}\NormalTok{,}\KeywordTok{sum}\NormalTok{(divmatrix[}\DecValTok{3}\NormalTok{,}\DecValTok{1}\NormalTok{:}\DecValTok{2}\NormalTok{])/}\DecValTok{2}\NormalTok{)}
\NormalTok{divmatrix2[,}\DecValTok{2}\NormalTok{] <-}\StringTok{ }\KeywordTok{c}\NormalTok{(}\KeywordTok{sum}\NormalTok{(divmatrix[}\DecValTok{3}\NormalTok{,}\DecValTok{1}\NormalTok{:}\DecValTok{2}\NormalTok{])/}\DecValTok{2}\NormalTok{,}\OtherTok{NA}\NormalTok{)}
\NormalTok{divmatrix2}
\end{Highlighting}
\end{Shaded}

\begin{verbatim}
##        A-B      C
## A-B     NA 0.0155
## C   0.0155     NA
\end{verbatim}

\begin{Shaded}
\begin{Highlighting}[]
\CommentTok{#find pair with minimum per site differences}
\KeywordTok{which}\NormalTok{(divmatrix2 ==}\StringTok{ }\KeywordTok{min}\NormalTok{(divmatrix2,}\DataTypeTok{na.rm=}\NormalTok{T),T)}
\end{Highlighting}
\end{Shaded}

\begin{verbatim}
##     row col
## C     2   1
## A-B   1   2
\end{verbatim}

\begin{Shaded}
\begin{Highlighting}[]
\CommentTok{#record number of per site differences}
\NormalTok{y2 <-}\StringTok{ }\KeywordTok{c}\NormalTok{(}\StringTok{"A-B-C"}\NormalTok{,}\KeywordTok{min}\NormalTok{(divmatrix2,}\DataTypeTok{na.rm=}\NormalTok{T))}

\NormalTok{y <-}\StringTok{ }\KeywordTok{rbind}\NormalTok{(y1,y2)}
\KeywordTok{colnames}\NormalTok{(y) <-}\StringTok{ }\KeywordTok{c}\NormalTok{(}\StringTok{"NeighborsJoined"}\NormalTok{,}\StringTok{"TiptoMRCADistance"}\NormalTok{)}
\KeywordTok{data.frame}\NormalTok{(y)}
\end{Highlighting}
\end{Shaded}

\begin{verbatim}
##    NeighborsJoined TiptoMRCADistance
## y1             A-B             0.006
## y2           A-B-C            0.0155
\end{verbatim}

\begin{figure}[H]
\centering
\includegraphics[width=0.8\textwidth]{fig3.jpg}
\caption{\label{fig:njtree}Neighbor Joining Tree based on data from Bryc et. al (2009).}
\end{figure}

\subsection{Given the divergence time between A and B and the number of
observed differences, what is the estimated mutation rate for these
species? (5
pts)}\label{given-the-divergence-time-between-a-and-b-and-the-number-of-observed-differences-what-is-the-estimated-mutation-rate-for-these-species-5-pts}

\begin{Shaded}
\begin{Highlighting}[]
\NormalTok{abdivergetime <-}\StringTok{ }\DecValTok{2000000}     \CommentTok{#divergence time between A and B in years}
\NormalTok{abdiff <-}\StringTok{ }\DecValTok{6}\NormalTok{/}\DecValTok{1000}             \CommentTok{#per site differences between A and B}
\NormalTok{acdiff <-}\StringTok{ }\DecValTok{14}\NormalTok{/}\DecValTok{1000}            \CommentTok{#per site differences between A and C}
\NormalTok{bcdiff <-}\StringTok{ }\DecValTok{17}\NormalTok{/}\DecValTok{1000}            \CommentTok{#per site differences between B and C}
\NormalTok{gtime <-}\StringTok{ }\DecValTok{3}                   \CommentTok{#generation time}


\NormalTok{abdivergegen <-}\StringTok{ }\NormalTok{abdivergetime/gtime     }\CommentTok{#divergence time between A and B in generations}

\NormalTok{mu =}\StringTok{ }\NormalTok{abdiff /}\StringTok{ }\NormalTok{(abdivergegen *}\StringTok{ }\DecValTok{2}\NormalTok{)        }
\NormalTok{mu}
\end{Highlighting}
\end{Shaded}

\begin{verbatim}
## [1] 4.5e-09
\end{verbatim}

Thus, the mutation rate is 4.5 x 10$^{-9}$ per site per generation
\#\#Assuming the same mutation rate for species C, what is the
divergence time between A and C? What about between B and C? (5 pts)

\begin{Shaded}
\begin{Highlighting}[]
\NormalTok{acdivergegen =}\StringTok{ }\NormalTok{acdiff /}\StringTok{ }\NormalTok{(}\DecValTok{2}\NormalTok{*mu)}
\NormalTok{acdivergetime =}\StringTok{ }\NormalTok{acdivergegen *}\StringTok{ }\NormalTok{gtime}
\NormalTok{acdivergetime}
\end{Highlighting}
\end{Shaded}

\begin{verbatim}
## [1] 4666667
\end{verbatim}

Thus, the divergence time between A and C is approximately 4.7 million
years.

\begin{Shaded}
\begin{Highlighting}[]
\NormalTok{bcdivergegen =}\StringTok{ }\NormalTok{bcdiff /}\StringTok{ }\NormalTok{(}\DecValTok{2}\NormalTok{*mu)}
\NormalTok{bcdivergetime =}\StringTok{ }\NormalTok{bcdivergegen *}\StringTok{ }\NormalTok{gtime}
\NormalTok{bcdivergetime}
\end{Highlighting}
\end{Shaded}

\begin{verbatim}
## [1] 5666667
\end{verbatim}

Thus, the divergence time between A and C is approximately 5.7 million
years. \#\#What is the estimated divergence time since the last common
ancestor of A, B, and C? (5 pts)

\begin{Shaded}
\begin{Highlighting}[]
\NormalTok{abcdivergegen =}\StringTok{ }\FloatTok{0.0155} \NormalTok{/}\StringTok{ }\NormalTok{(}\DecValTok{2}\NormalTok{*mu)}
\NormalTok{abcdivergetime =}\StringTok{ }\NormalTok{abcdivergegen *}\StringTok{ }\NormalTok{gtime}
\NormalTok{abcdivergetime}
\end{Highlighting}
\end{Shaded}

\begin{verbatim}
## [1] 5166667
\end{verbatim}

Thus, the divergence time since the last common ancestor between A, B,
and C is approximately 5.2 million years.

\end{document}
